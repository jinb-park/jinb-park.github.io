%%%%%%%%%%%%%%%%%%%%%%%%%%%%%%%%%%%%%%%%%%%%%%%%%%%%%%%%%%%%%%%%%%%%%%%%%%%%%%%
% A clean template for an academic CV
%
% Uses tabularx to create two column entries (date and job/edu/citation).
% Defines commands to make adding entries simpler.
%
%%%%%%%%%%%%%%%%%%%%%%%%%%%%%%%%%%%%%%%%%%%%%%%%%%%%%%%%%%%%%%%%%%%%%%%%%%%%%%%

\documentclass[10pt, a4paper]{article}

% Full Unicode support for non-ASCII characters
\usepackage[utf8]{inputenc}

% Useful aliases
\newcommand{\UERJ}{Universidade do Estado do Rio de Janeiro}
\newcommand{\UHM}{University of Hawai`i at M\={a}noa}
\newcommand{\SOEST}{School of Ocean and Earth Science and Technology}
\newcommand{\UHEARTH}{Department of Earth Sciences}
\newcommand{\LIVEARTH}{Department of Earth, Ocean and Ecological Sciences}
\newcommand{\LIVENV}{School of Environmental Sciences}
\newcommand{\LIV}{University of Liverpool}

% Identifying information
\newcommand{\Title}{Curriculum Vit\ae}
\newcommand{\FirstName}{Jinbum}
\newcommand{\LastName}{Park}
\newcommand{\Initials}{L}
\newcommand{\MyName}{\FirstName\ \LastName}
\newcommand{\Me}{\textbf{\LastName, \Initials}}  % For citations
\newcommand{\Email}{jinb.park7@gmail.com}
\newcommand{\PersonalWebsite}{https://jinb-park.github.io}
%\newcommand{\LabWebsite}{www.compgeolab.org}
%\newcommand{\ORCID}{0000-0001-6123-9515}
\newcommand{\Address}{
  %Jane Herdman Building \\ 4 Brownlow Street \\ Liverpool, L69 3GP \\ United Kingdom
  South Korea
}

% Names for citing coauthors
\newcommand{\Val}{Barbosa, VCF}
\newcommand{\Bi}{Oliveira Jr, VC}
\newcommand{\Paul}{Wessel, P}
\newcommand{\Joaquim}{Luis, J}
\newcommand{\Remko}{Scharroo, R}
\newcommand{\Florian}{Wobbe, F}
\newcommand{\Walter}{Smith, WHF}
\newcommand{\Dongdong}{Tian, D}
\newcommand{\Bridget}{Smith-Konter, B}
\newcommand{\Eric}{Xu, X}
\newcommand{\David}{Sandwell, DT}
\newcommand{\Carla}{Braitenberg, C}
\newcommand{\Naomi}{Ussami, N}
\newcommand{\Manoel}{D'Agrella-Filho, MS}
\newcommand{\JB}{Silva, JBC}
\newcommand{\Dai}{Sales, DP}
\newcommand{\Figura}{Melo, FF}
\newcommand{\Dio}{Carlos, DU}
\newcommand{\BragaVale}{Braga, MA}
\newcommand{\YLi}{Li, Y}
\newcommand{\Angeli}{Angeli, G}
\newcommand{\Peres}{Peres, G}
\newcommand{\Everton}{Bomfim, EP}
\newcommand{\Eder}{Molina, E}
\newcommand{\Gomes}{Gomes, AAS}
\newcommand{\Santiago}{Soler, SR}
\newcommand{\Agustina}{Pesce, A}
\newcommand{\Gimenez}{Gimenez, ME}
\newcommand{\Kristoffer}{Hallam, KAT}
\newcommand{\Guangdong}{Zhao, G}
\newcommand{\Bo}{Chen, B}
\newcommand{\JLiu}{Liu, J}
\newcommand{\LChen}{Chen, L}
\newcommand{\RGuo}{Guo, R}
\newcommand{\MKaban}{Kaban, MK}
\newcommand{\Lindsey}{Heagy, LJ}
\newcommand{\Lion}{Krischer, L}
\newcommand{\Rene}{Gassmoeller, R}
\newcommand{\Bane}{Sullivan, CB}
\newcommand{\Jens}{Klump, JF}
\newcommand{\LBarba}{Barba, LA}
\newcommand{\JBazan}{Bazan, J}
\newcommand{\JBrown}{Brown, J}
\newcommand{\RGuimera}{Guimera, RV}
\newcommand{\MGymrek}{Gymrek, M}
\newcommand{\AHanna}{Alex Hanna}
\newcommand{\KHuff}{Huff, KD}
\newcommand{\DKatz}{Katz, DS}
\newcommand{\CMadan}{Madan, CR}
\newcommand{\KMoerman}{Moerman, KM}
\newcommand{\KNiemeyer}{Niemeyer, KE}
\newcommand{\JPoulson}{Poulson, JL}
\newcommand{\PPrins}{Prins, P}
\newcommand{\KRam}{Ram, K}
\newcommand{\ARokem}{Rokem, A}
\newcommand{\Arfon}{Smith, AM}
\newcommand{\GThiruvathukal}{Thiruvathukal, GK}
\newcommand{\KThyng}{Thyng, KM}
\newcommand{\BWilson}{Wilson, BE}
\newcommand{\Yehudi}{Yehudi, Y}
\newcommand{\Remi}{Rampin, R}
\newcommand{\Hugo}{van Kemenade, H}
\newcommand{\MattTurk}{Turk, M}
\newcommand{\Shapero}{Shapero, D}
\newcommand{\Anderson}{Banihirwe, A}
\newcommand{\Leeman}{Leeman, J}
\newcommand{\JEbbing}{Ebbing, J}
\newcommand{\AGuy}{Guy, A}
\newcommand{\JFarquharson}{Farquharson, J}
\newcommand{\AKushnir}{Kushnir, A}
\newcommand{\FWadsworth}{Wadsworth, F}
\newcommand{\LPerozzi}{Perozzi, L}
\newcommand{\MWieczorek}{Wieczorek, MA}
\newcommand{\LLi}{Li, L}
\newcommand{\Ricardo}{Trindade, RIF}


% Template configuration
%%%%%%%%%%%%%%%%%%%%%%%%%%%%%%%%%%%%%%%%%%%%%%%%%%%%%%%%%%%%%%%%%%%%%%%%%%%%%%%

% Disable hyphenation
\usepackage[none]{hyphenat}

% Control the font size
\usepackage{anyfontsize}

% Icon fonts (requires using xelatex or luatex)
\usepackage[fixed]{fontawesome5}
\usepackage{academicons}

% Template variables for styling
\newcommand{\TablePad}{\vspace{-0.4cm}}
\newcommand{\SoftwareTitle}[1]{{\bfseries #1}}
\newcommand{\TableTitle}[1]{{\fontsize{12pt}{0}\selectfont \itshape #1}}

% For fancy and multipage tables
\usepackage{tabularx}
\usepackage{ltablex}

% Define a new environment to place all CV entries in a 2-column table.
% Left column are the dates, right column the entries.
\usepackage{environ}
\NewEnviron{EntriesTable}{
\TablePad
\begin{tabularx}{\textwidth}{@{}p{0.10\textwidth}@{\hspace{0.02\textwidth}}p{0.88\textwidth}@{}}
  \BODY
\end{tabularx}
}
\NewEnviron{EntriesTableExtra}{
\TablePad
\begin{tabularx}{\textwidth}{@{}p{0.10\textwidth}@{\hspace{0.02\textwidth}}p{0.79\textwidth}@{\hspace{0.02\textwidth}}>{\raggedright\arraybackslash}p{0.07\textwidth}}
  \BODY
\end{tabularx}
}

% Macros to add links and mark publications
\newcommand{\DOI}[1]{doi:\href{https://doi.org/#1}{#1}}
\newcommand{\DOILink}[1]{\href{https://doi.org/#1}{doi.org/#1}}
\newcommand{\Website}[1]{\href{https://#1}{#1}}
\newcommand{\Preprint}[1]{\href{https://doi.org/#1}{\faFilePdf}}
\newcommand{\Paper}[1]{\href{#1}{\faFilePdf}}
\newcommand{\Youtube}[1]{\href{https://www.youtube.com/watch?v=#1}{\faYoutube}}
\newcommand{\GitHub}[1]{\href{https://github.com/#1}{\faGithub}}
\newcommand{\GitHubPage}[1]{\href{#1}{\faGithub}}
\newcommand{\Data}[1]{\href{https://doi.org/#1}{\faChartLine}}
\newcommand{\Slides}[1]{\href{https://#1}{\faTv}}
\newcommand{\Video}[1]{\href{#1}{\faYoutube}}
\newcommand{\SlidesDOI}[1]{\href{https://doi.org/#1}{\faTv}}
\newcommand{\PosterDOI}[1]{\href{https://doi.org/#1}{\faImage}}
\newcommand{\OA}{\thinspace\aiOpenAccess\enspace}

% Macros to set the year and duration on the left column
\newcommand{\Duration}[2]{\fontsize{9pt}{0}\selectfont #1 -- #2}
\newcommand{\Year}[1]{\fontsize{9pt}{0}\selectfont #1}
\newcommand{\Ongoing}{on}
\newcommand{\Future}{future}
\newcommand{\Appointment}[4]{\textbf{#1} \newline #2 \newline #3 \newline #4}

% Define command to insert month name and year as date
\usepackage{datetime}
\newdateformat{monthyear}{\monthname[\THEMONTH], \THEYEAR}

% Set the page margins
\usepackage[a4paper,margin=1.5cm,includehead,headsep=5mm]{geometry}

% To get the total page numbers (\pageref{LastPage})
\usepackage{lastpage}

% No indentation
\setlength\parindent{0cm}

% Increase the line spacing
\renewcommand{\baselinestretch}{1.2}
% and the spacing between rows in tables
\renewcommand{\arraystretch}{1.5}

% Remove space between items in itemize and enumerate
\usepackage{enumitem}
\setlist{nosep}

% Use custom colors
\usepackage[usenames,dvipsnames]{xcolor}

% Set fonts (requires compilation with xelatex)
\usepackage{fontspec}
\setmainfont[%
  Path = fonts/notoserif/,
  UprightFont = NotoSerif-Regular,
  BoldFont = NotoSerif-Bold,
  ItalicFont = NotoSerif-Italic,
  Extension = .ttf
]{NotoSerif}



% Set the spacing for sections
\usepackage{titlesec}
\titleformat{\section}
  {\normalfont\Large\mdseries} % format
  {} % label
  {0pt} % separation (left separation for hang)
  {} % text before title
  [\titlerule] % text after title
\titleformat{\subsection}
  {\normalfont\large\mdseries} % format
  {} % label
  {0pt} % separation (left separation for hang)
  {} % text before title

% Disable number of sections. Use this instead of "section*" so that the sections still
% appear as PDF bookmarks. Otherwise, would have to add the table of contents entries
% manually.
\makeatletter
\renewcommand{\@seccntformat}[1]{}
\makeatother

% Set fancy headers
\usepackage{fancyhdr}
\pagestyle{fancy}
\fancyhf{}
%\lhead{\fontsize{9pt}{10pt}\selectfont
%  \monthyear\today
%}
\lhead{
  \fontsize{9pt}{10pt}\selectfont
  \MyName
  \hspace{0.2cm} -- \hspace{0.2cm}
  \Title
}
\rhead{\fontsize{9pt}{10pt}\selectfont \thepage{} of \pageref*{LastPage}}
\renewcommand{\headrulewidth}{0pt}

% Metadata for the PDF output and control of hyperlinks
\usepackage[colorlinks=true]{hyperref}
\hypersetup{
  pdftitle={\MyName\ - \Title},
  pdfauthor={\MyName},
  linkcolor=blue,
  citecolor=blue,
  filecolor=black,
  urlcolor=MidnightBlue
}
%%%%%%%%%%%%%%%%%%%%%%%%%%%%%%%%%%%%%%%%%%%%%%%%%%%%%%%%%%%%%%%%%%%%%%%%%%%%%%%


\begin{document}

% No header for the first page
\thispagestyle{empty}

%%%%%%%%%%%%%%%%%%%%%%%%%%%%%%%%%%%%%%%%%%%%%%%%%%%%%%%%%%%%%%%%%%%%%%%%%%%%%%%
\begin{minipage}[t]{0.7\textwidth}
{\fontsize{22pt}{0}\selectfont\MyName}
\end{minipage}
%\begin{minipage}[t]{0.3\textwidth}
%  \begin{flushright}
%    Last updated: \monthyear\today
%  \end{flushright}
%\end{minipage}
\\[-0.1cm]
\rule{\textwidth}{2pt}
\\[0.1cm]
\begin{minipage}[t]{0.7\textwidth}
    Email: \href{mailto:\Email}{\Email}
    \\
    Affiliation: \href{https://research.samsung.com/security_privacy}{Samsung Research, Security \& Privacy Team}
    \\
    Website: \Website{\PersonalWebsite}
    \\
    Blog: \Website{https://jinb-park.github.io/blog}
    \\
    Google Scholar: \href{https://scholar.google.com/citations?user=e-o2O2IAAAAJ}{https://scholar.google.com/citations?user=e-o2O2IAAAAJ}
\end{minipage}
\begin{minipage}[t]{0.3\textwidth}
  \begin{flushright}
    \Address
  \end{flushright}
\end{minipage}

%%%%%%%%%%%%%%%%%%%%%%%%%%%%%%%%%%%%%%%%%%%%%%%%%%%%%%%%%%%%%%%%%%%%%%%%%%%%%%%
\section{Research Interest}

\begin{itemize}
  \item Trusted execution environments (TrustZone, ARM CCA, SGX, Secure processors)
  \item Confidential computing
  \item OS kernel security
  \item Side-channel attacks and defenses
  \item Bug finding and exploitations
  \item Machine learning security (e.g., federated learning, LLM security)
  \item Applied cryptography (e.g., Zero Knowledge Proof)
\end{itemize}
%%%%%%%%%%%%%%%%%%%%%%%%%%%%%%%%%%%%%%%%%%%%%%%%%%%%%%%%%%%%%%%%%%%%%%%%%%%%%%%

%%%%%%%%%%%%%%%%%%%%%%%%%%%%%%%%%%%%%%%%%%%%%%%%%%%%%%%%%%%%%%%%%%%%%%%%%%%%%%%
\section{Education}

\begin{EntriesTable}
  \Duration{2006}{2013}  &
  \textbf{BSc in Department of Software}, Gachon University, South Korea
\end{EntriesTable}
%%%%%%%%%%%%%%%%%%%%%%%%%%%%%%%%%%%%%%%%%%%%%%%%%%%%%%%%%%%%%%%%%%%%%%%%%%%%%%%

%%%%%%%%%%%%%%%%%%%%%%%%%%%%%%%%%%%%%%%%%%%%%%%%%%%%%%%%%%%%%%%%%%%%%%%%%%%%%%%
\section{Projects}

All projects listed below were done in Samsung Research.
\newline

\begin{EntriesTable}
  \Duration{2023.04}{\Ongoing}  &
  \textbf{Islet: An on-device confidential computing platform}
  \begin{itemize}
    \item \textbf{Role.} A developer and researcher
    \item \textbf{Type.} Open source project (\href{https://confidentialcomputing.io/projects/current-projects/}{an official Confidential Computing Consortium (CCC) project})
    \item Developing a whole software stack, fully written in Rust, to power ARM CCA. (based on the ARM CCA specification)
    \item Developed an end-to-end confidential AI demo scenario (for details, see \href{https://github.com/islet-project/islet/tree/main/examples/confidential-ml}{here}).
    \item Implemented an integration with \href{https://github.com/ccc-certifier-framework/certifier-framework-for-confidential-computing}{the certifier framework} to build an end-to-end heterogeneous CC (Confidential Computing) protection.
    \item An academic research towards privacy-preserving CC framework (work in progress internally as a leading author).
  \end{itemize}
  \\
  \Duration{2022.05}{2023.04}  &
  a period of time for parental leave
  \\
  \Duration{2021}{2022}  &
  \textbf{A federated learning framework for mobile devices}
  \begin{itemize}
    \item \textbf{Role.} Lead developer
    \item \textbf{Type.} Proof-of-concept project (not deployed in production)
    \item Developed an android based (Java) on-device federated learning framework built on top of a TensorFlowLite library modified to be able to do training on devices.
    \item Developed a federated learning server (Python) that communicates with devices through gRPC.
    \item Did a field test with 20 android devices on a location-based service deep learning model.
  \end{itemize}
  \\
  \Duration{2020}{2021}  &
  \textbf{Rust-based full-stack OS for secure processor}
  \begin{itemize}
    \item \textbf{Role.} Lead kernel developer and one of the application layer developers
    \item \textbf{Type.} In development while aiming to be in production (but not yet released)
    \item Developed a Rust-based kernel from scratch, which targets ARM Cortex-M boards and doesn't rely on Rust's std library.
    \item Developed an application layer (a set of system calls and libraries) and an async backend that allows applications to use Rust's async capability.
  \end{itemize}
  \\
  \Duration{2019}{2020}  &
  \textbf{A TrustZone-based secure enclave}
  \begin{itemize}
    \item \textbf{Role.} Lead developer (one-man project)
    \item \textbf{Type.} Proof-of-concept project (not deployed in production)
    \item Designed and developed an SGX-like enclave architecture on top of ARM TrustZone, thereby allowing mobile developers to take SGX's programming model. (Rust and C++)
    \item Developed a new small Rust compiler toolchain for this architecture.
  \end{itemize}
  \\
  \Duration{2018}{2019}  &
  \textbf{A real-time kernel protection}
  \begin{itemize}
    \item \textbf{Role.} One of the core developers
    \item \textbf{Type.} Developed for autonomous platforms but not deployed
    \item Designed and developed a Type-1 hypervisor on ARMv8-A, which ensures that Linux's non-writable memory regions are not corrupted. This is similar in concept to \href{https://www.samsungknox.com/en/blog/real-time-kernel-protection-rkp}{KNOX RKP} in galaxy devices.
    \item Written in C and ARM assembly.
  \end{itemize}
  \\
  \Duration{2014}{2017}  &
  \textbf{System Integrity Monitor (SIM) version 1.0--3.0}
  \begin{itemize}
    \item \textbf{Role.} Lead developer
    \item \textbf{Type.} Deployed as the key part of \href{https://news.samsung.com/global/samsung-electronics-announces-gaia-a-powerful-smart-tv-security-solution-for-2016-and-beyond}{GAIA} which is Samsung SMART TV's security solution.
    \item Designed and developed a Linux kernel monitoring system that utilizes ARM TrustZone and a proprietary memory bus snooping system.\
      It aims to prevent and detect corruptions on non-writable memory regions and security-critical kernel read-write data. Also, it plays a crucial role in the secure boot and attestations of Samsung SMART TVs. (C and C++)
    \item Developed device drivers for Linux kernel and TrustZone secure kernel. (C and ARM assembly)
    \item Developed a daemon service that runs as a system service of the Tizen TV platform and takes local/remote attestation requests from other processes. (C++)
    \item Designed PKI (Public Key Infrastructure) and cryptographic key operations for this system.
    \item Designed attestation servers and supported server developers.
  \end{itemize}
  \\
  \Duration{2013}{2014}  &
  \textbf{Samsung DRM (SDRM)}
  \begin{itemize}
    \item \textbf{Role.} Associate developer
    \item \textbf{Type.} Deployed in Samsung SMART TVs to protect 4k contents.
    \item Migrated the existing SDRM codes into ARM TrustZone. (C and C++)
    \item Developed the SDRM media plugin for the Tizen TV platform.
    \item Managed PKI (Public Key Infrastructure) and cryptographic key operations for this system.
  \end{itemize}
\end{EntriesTable}

%%%%%%%%%%%%%%%%%%%%%%%%%%%%%%%%%%%%%%%%%%%%%%%%%%%%%%%%%%%%%%%%%%%%%%%%%%%%%%%
\section{Publications}

\begin{EntriesTable}
  \Year{2025} &
  EdgeShield: A Security Monitor Framework For On-Device Confidential Computing
  \begin{itemize}
    \item \textbf{Jinbum Park}, Bokdeuk Jeong, Sunwook Eom, Taesoo Kim
    \item Under submission
  \end{itemize}
  \\
  \Year{2025} &
  TikTag: Breaking ARM’s Memory Tagging Extension with Speculative Execution \Paper{https://arxiv.org/abs/2406.08719}
  \begin{itemize}
    \item Juhee Kim, \textbf{Jinbum Park}, Sihyeon Roh, Jaeyoung Chung, Youngjoo Lee, Taesoo Kim, Byoungyoung Lee
    \item IEEE Security and Privacy 2025 (\emph{top-tier conference}) (to appear)
  \end{itemize}
  \\
  \Year{2024} &
  PeTAL: Ensuring Access Control Integrity against Data-only Attacks on Linux \Paper{https://www.cs.ucr.edu/~csong/ccs24-petal.pdf}
  \begin{itemize}
    \item Juhee Kim, \textbf{Jinbum Park}, Yoochan Lee, Chengyu Song, Taesoo Kim, Byoungyoung Lee
    \item ACM CCS 2024 (\emph{top-tier conference})
  \end{itemize}
  \\
  \Year{2022} &
  In-Kernel Control-Flow Integrity on Commodity OSes using ARM Pointer Authentication \Paper{https://www.usenix.org/system/files/sec22-yoo.pdf} \GitHub{SamsungLabs/PALinux}
  \begin{itemize}
    \item Sungbae Yoo(*), \textbf{Jinbum Park(*)}, Seolheui Kim, Yeji Kim, Taesoo Kim (*: co-leading authors)
    \item The 31st USENIX Security Symposium (USENIX Security 2022) (\emph{top-tier conference})
  \end{itemize}
  \\
  \Year{2022} &
  ViK: Practical Mitigation of Temporal Memory Safety Violations through Object ID Inspection \Paper{https://dl.acm.org/doi/10.1145/3503222.3507780}
  \begin{itemize}
    \item Haehyun Cho, \textbf{Jinbum Park}, Adam Oest, Tiffany Bao, Ruoyu Wang, Yan Shoshitaishvili, Adam Doupé, Gail-Joon Ahn
    \item The 27th ACM International Conference on Architectural Support for Programming Languages and Operating Systems (ASPLOS ‘22) (\emph{top-tier conference})
  \end{itemize}
  \\
  \Year{2020} &
  Exploiting Uses of Uninitialized Stack Variables in Linux Kernels to Leak Kernel Pointers \Paper{https://www.usenix.org/system/files/woot20-paper-cho.pdf} \GitHub{jinb-park/leak-kptr}
  \begin{itemize}
    \item Haehyun Cho, \textbf{Jinbum Park}, Joonwon Kang, Tiffany Bao, Ruoyu Wang, Yan Shoshitaishvili, Adam Doupe, Gail-Joon Ahn
    \item The 14th USENIX Workshop on Offensive Technologies (WOOT ‘20)
  \end{itemize}
  \\
  \Year{2020} &
  SmokeBomb: Effective Mitigation Method against Cache Side-channel Attacks on the ARM Architecture \Paper{https://dl.acm.org/doi/pdf/10.1145/3386901.3388888} \GitHub{SamsungLabs/smoke-bomb}
  \begin{itemize}
    \item Haehyun Cho, \textbf{Jinbum Park}, Donguk Kim, Ziming Zhao, Yan Shoshitaishvili, Adam Doupe, Gail-Joon Ahn
    \item The 18th ACM International Conference on Mobile Systems, Applications, and Services (MobiSys 2020) (\emph{top-tier conference})
  \end{itemize}
  \\
  \Year{2018} &
  Prime+Count: Novel Cross-world Covert Channels on ARM TrustZone \Paper{https://dl.acm.org/doi/10.1145/3274694.3274704} \GitHub{SamsungLabs/prime-count}
  \begin{itemize}
    \item Haehyun Cho, Penghui Zhang, Donguk Kim, \textbf{Jinbum Park}, Choong-Hoon Lee, Ziming Zhao, Adam Doupé, and Gail-Joon Ahn
    \item Annual Computer Security Applications Conference (ACSAC) 2018
  \end{itemize}
  \\
  \Year{2016} &
  A Snoop-Based Kernel Introspection System against Address Translation Redirection Attack
  \begin{itemize}
    \item Donguk Kim, Jihoon Kim, \textbf{Jinbum Park}, Jinmok Kim
    \item Journal of The Korea Institute of Information Security \& Cryptology VOL.26, NO.5, Oct. 2016
  \end{itemize}
  \\
  \Year{2015} &
  An Efficient Kernel Introspection System using a Secure Timer on TrustZone
  \begin{itemize}
    \item Jinmok Kim, Donguk Kim, \textbf{Jinbum Park}, Jihoon Kim, Hyoungshick Kim
    \item Journal of The Korea Institute of Information Security \& Cryptology VOL.25, NO.4, Aug. 2015
  \end{itemize}

\end{EntriesTable}
%%%%%%%%%%%%%%%%%%%%%%%%%%%%%%%%%%%%%%%%%%%%%%%%%%%%%%%%%%%%%%%%%%%%%%%%%%%%%%%

%%%%%%%%%%%%%%%%%%%%%%%%%%%%%%%%%%%%%%%%%%%%%%%%%%%%%%%%%%%%%%%%%%%%%%%%%%%%%%%
\section{Talks (industry conferences)}

\begin{EntriesTable}
  \Year{2024} &
  On-Device Confidential Computing: Updates on Our Activities and Future Potential \Video{https://youtu.be/YP2jFD9uPjk?si=Kn-K_ysCAf9K_SDA}
  \begin{itemize}
    \item \textbf{Jinbum Park, Heeill Wang}
    \item Samsung Security Tech Forum 2024 (SSTF 2024)
  \end{itemize}
  \\
  \Year{2024} &
  Breaking ARM MTE with Speculative Execution
  \begin{itemize}
    \item \textbf{Jinbum Park}
    \item Zer0Con 2024
  \end{itemize}
  \\
  \Year{2022} &
  Taking Kernel Hardening to the Next Level \Slides{i.blackhat.com/Asia-22/Friday-Materials/AS-22-Park-Taking-Kernel-Hardening-to-the-Next-Level.pdf} \Video{https://www.youtube.com/watch?v=1titzBiuxSc}
  \begin{itemize}
    \item \textbf{Jinbum Park}, Haehyun Cho, Sungbae Yoo, Seolheui Kim, Yeji Kim, Bumhan Kim, Taesoo Kim
    \item Blackhat ASIA 2022
  \end{itemize}
  \\
  \Year{2020} &
  Cache Attacks on Various CPU Architectures \Slides{jinb-park.github.io/cache-attack-poc2020-r2.pdf} \Video{https://drive.google.com/file/d/1sqasfokB0LkGUvpo_G-z0XNODu4EJkJM/view}
  \begin{itemize}
    \item \textbf{Jinbum Park}
    \item POC 2020
  \end{itemize}
  \\
  \Year{2019} &
  Micro-architectural attack and defense on Linux kernel \Slides{www.ssdc.kr/content/data/session/Day\%201_1630_2.pdf}
  \begin{itemize}
    \item \textbf{Jinbum Park}, Joonwon Kang
    \item Samsung Open Source Conference (SOSCON) 2019
  \end{itemize}
  \\
  \Year{2019} &
  Leak kernel pointer by exploiting uninitialized uses in Linux kernel \Slides{jinb-park.github.io/leak-kptr.pdf} \GitHub{jinb-park/leak-kptr}
  \begin{itemize}
    \item \textbf{Jinbum Park}
    \item Zer0Con 2019
  \end{itemize}
  \\
  \Year{2018} &
  Attack and Defense on Linux kernel \GitHub{jinb-park/linux-exploit}
  \begin{itemize}
    \item \textbf{Jinbum Park}
    \item Samsung Open Source Conference (SOSCON) 2018
  \end{itemize}
  \\
  \Year{2018} &
  Exploit Linux kernel eBPF with side-channel \Slides{jinb-park.github.io/Exploit-Linux-kernel-eBPF-with-side-channel.html} \GitHub{jinb-park/linux-exploit}
  \begin{itemize}
    \item \textbf{Jinbum Park}
    \item KIMCHICON 2018
  \end{itemize}
\end{EntriesTable}
%%%%%%%%%%%%%%%%%%%%%%%%%%%%%%%%%%%%%%%%%%%%%%%%%%%%%%%%%%%%%%%%%%%%%%%%%%%%%%%

%%%%%%%%%%%%%%%%%%%%%%%%%%%%%%%%%%%%%%%%%%%%%%%%%%%%%%%%%%%%%%%%%%%%%%%%%%%%%%%
\section{Open sources}

\begin{EntriesTable}
  \Year{-} &
  KSPP Study: Analysis on Kernel Self-Protection: Understanding Security and Performance Implication \GitHubPage{https://samsung.github.io/kspp-study/}
  \begin{itemize}
    \item Analyzed security and performance analysis for kernel self-protection projects
  \end{itemize}
  \\
  \Year{-} &
  CSCA: Crypto Side Channel Attack \GitHub{jinb-park/crypto-side-channel-attack}
  \begin{itemize}
    \item Developed cache-based crypto side-channel attacks on both x86\_64 and ARM64 (e.g., recovering a full AES-128 key)
  \end{itemize}
  \\
  \Year{-} &
  Linux kernel contributions (selected)
  \begin{itemize}
    \item Fix vulnerable gadgets to spectre-variant1 attack (patch \href{https://git.kernel.org/pub/scm/linux/kernel/git/torvalds/linux.git/commit/?id=55690c07b44a}{0},\href{https://git.kernel.org/pub/scm/linux/kernel/git/torvalds/linux.git/commit/?id=3a2af7cccbba}{1})
    \item arm: Makes ptdump reusable and add WX page checking (\href{https://lkml.org/lkml/2017/12/7/321}{patch})
    \item arm: Add ARCH\_HAS\_FORTIFY\_SOURCE (patch \href{https://git.kernel.org/pub/scm/linux/kernel/git/torvalds/linux.git/commit/?id=73b9160d0dfe}{0},\href{https://git.kernel.org/pub/scm/linux/kernel/git/torvalds/linux.git/commit/?id=ee333554fed5}{1})
  \end{itemize}
  \\
  \Year{-} &
  Ubuntu kernel contributions
  \begin{itemize}
    \item Revert barrier-patch which turns out be vulnerable to variant4 attack (patch \href{https://git.launchpad.net/~ubuntu-kernel/ubuntu/+source/linux/+git/xenial/commit/?id=cb0321f01227}{0},\href{https://git.launchpad.net/~ubuntu-kernel/ubuntu/+source/linux/+git/xenial/commit/?id=48a028480eb0}{1})
  \end{itemize}
\end{EntriesTable}
%%%%%%%%%%%%%%%%%%%%%%%%%%%%%%%%%%%%%%%%%%%%%%%%%%%%%%%%%%%%%%%%%%%%%%%%%%%%%%%

%%%%%%%%%%%%%%%%%%%%%%%%%%%%%%%%%%%%%%%%%%%%%%%%%%%%%%%%%%%%%%%%%%%%%%%%%%%%%%%
\section{Skills}

\textbf{Languages.} 
\begin{itemize}
  \item Korean, English
\end{itemize}

\textbf{Programming Languages.}
\begin{itemize}
  \item C, C++, Python, Rust, Assembly (x86\_64 and ARM)
\end{itemize}

\textbf{Hardware.}
\begin{itemize}
  \item ARM: ARM Cortex-A, ARM Cortex-M, ARM TrustZone, ARM CCA, ARM pointer authentication, ARM memory tagging extension
  \item Intel: x86\_64, SGX
  \item Developed several security-relevant arch-specific codes and cache attacks/defenses on both architectures.
\end{itemize}

\textbf{Low-level software.}
\begin{itemize}
  \item Kernel: Linux, FreeBSD
  \item Hypervisor: KVM, a light-weight security monitor (e.g., RMM in ARM CCA)
\end{itemize}

\textbf{Compiler.}
\begin{itemize}
  \item LLVM, GCC (developed several static analysis passes on LLVM and GCC)
\end{itemize}

\textbf{Domain knowledge.}
\begin{itemize}
  \item System and software security 
  \item Operating system kernel and hardware architectures
  \item Offensive techniques (kernel exploits and bug findings)
  \item Mobile platforms (Tizen and Android)
  \item Applied cryptography
  \item Machine learning and deep learning
  \item Zero-Knowledge Proof
\end{itemize}

%%%%%%%%%%%%%%%%%%%%%%%%%%%%%%%%%%%%%%%%%%%%%%%%%%%%%%%%%%%%%%%%%%%%%%%%%%%%%%%

\end{document}
